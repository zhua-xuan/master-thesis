\chapter{Introduction}\label{chapter:introduction}
% add the financial market is like, the purpose is to find...
% [Main idea: Context - Modern trading challenges for large institutional orders]
Trading in financial markets has become increasingly automated and high-frequent, especially foreign exchange markets. When institutional investors like pension funds want to buy or sell large amounts of currency, they cannot simply place one big order at once. This would cause large market impact and cost a lot of money. Instead, they split large orders into many smaller ones and execute them over time using trading algorithms.

%[Main idea: Problem - Current backtesting systems are too simplistic]
To test these trading strategies before using them with real money, investors use backtesting. Backtesting means running the strategy on historical market data to see how it would have performed. However, current backtesting systems have a major problem. They use simple rules to decide whether orders get filled. For example, some methods rely on simple assumptions, such as filling possibilities based on the static bid-ask spread at each timestamp or using basic queue rules like time-priority and price-priority. However, these approaches do not capture the real dynamics of how trades happen in real markets.

%[Main idea: Gap - Missing aggressive trade prediction makes backtesting unrealistic]
In real markets, there come aggressive traders taking orders out of the order book. These aggressive trades are not random. Aggressive trades cluster together in time and depend on market conditions. But current backtesting systems cannot mimic the reality and predict when these aggressive trades will happen. These systems make costly mistakes in both filling rates and time costs. As a result, fewer orders are filled compared to real markets, and some may never be filled at all in high-frequency trading scenarios. This makes the backtesting results unrealistic and less useful for strategy development. Moreover, it leads to higher trading costs and lower profits for institutional investors. This thesis focuses on Foreign Exchange markets, because for pension funds managing billions of euros, even small improvements in execution can save millions in trading costs. 

%[Main idea: Scientific importance - Understanding market microstructure and timing patterns]
From a scientific standpoint, predicting aggressive trades helps us understand how financial markets really work. By studying aggressive trade patterns, researchers can build better models of trader behaviors and price dynamics. This knowledge is valuable for regulators who want to monitor market trends and for academics studying market microstructure. Additionally, when framed as a classification problem between aggressive and non-aggressive trades, the order book flow often shows severe class imbalance. It is therefore important to explore how model-based solutions can help address this issue.

%[Main idea: Literature gap - Existing approaches have limitations]
Previous research has tried to model order book using different approaches. Some studies use traditional statistical models, but these often assume fixed patterns that do not change over time. Other studies use machine learning methods, but these are hard to interpret and may not capture the timing patterns that are important in trading. 

%[Main idea: Research objective - Predict aggressive trades for better backtesting]
This thesis addresses the problem of predicting aggressive trades in foreign exchange (FX) markets. The goal is to create a more realistic and dynamic backtesting environment by predicting when aggressive traders will take passive orders out of the order flow. This is important because it helps trading algorithms better estimate when their orders will be filled and make smarter decisions about order timing and pricing. This leads to better execution performance and higher profits.

%[Main idea: Methodology - Three-part hybrid approach]
The approach of this thesis contains two stages. First, an XGBoost classifier handles the imbalance-problem that aggressive trades are very rare in the huge order flow. Second, a neural network captures embedded market features over time and passes the kernel to a Hawkes process. The result shows how aggressive trades are influenced by market conditions, how they cluster together and influence each other. This combination provides both dynamic predictions and clear explanations of why aggressive trades happen.

%[Main idea: Contribution - Dynamic backtesting framework]
The main contribution is a new framework that makes backtesting more realistic and dynamic. It originates from a practical problem faced by MN, a pension fund service provider managing €150 billion in assets, and aims to bridge this real-world challenge with an academic contribution in the area of predictive modeling. Instead of using fixed filling probabilities, the new framework predicts aggressive trade patterns based on actual market conditions and temporal dependencies. This helps traders develop better strategies, reduce unexpected losses from poor execution, and understand the true performance of their trading approaches. More generally, the framework is suitable for classification tasks where rare events, such as aggressive trades, not only occur infrequently but also influence the likelihood of future similar events. The model takes into account both the practical context and the temporal dependency between such events. This is why the framework can be applied to any high-frequency trading environment where execution quality is important, especially in settings with severe class imbalance. 

% how the rest of the thesis is set up

The rest of the thesis will proceed as follows. In Chapter~\ref{chapter:preliminary}, I conduct a preliminary analysis, including the benchmarking of current backtesting methods and a details data sources from the FX spot market. Chapter~\ref{chapter:methodology} is the most important part and presents the core methodology of this study. It introduces the definition and modeling of aggressive trades, the data processing pipeline, and a framework that integrates XGBoost and GRU-based Neural Hawkes Process. In Chapter~\ref{chapter:experiments}, I describe the empirical experiments based on certain trading days. It includes model training and the evaluation of predictions. Chapter~\ref{chapter:cd} introduces a discussion of the outcomes, the implications for FX backtesting environments, the potential for generalization to other financial instruments and possible improvement.