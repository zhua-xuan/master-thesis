\chapter{Conclusions and Discussion}\label{chapter:cd}
% [Main Findings and Contributions]
This thesis addressed a fundamental problem in quantitative finance: how to predict aggressive trades in foreign exchange markets to improve backtesting environments. As discussed in the introduction and business context, current backtesting systems rely on oversimplified rules that fail to capture the complex dynamics of real markets. The automatic FX trading platform at MN is a good idea and the application starting point of our research. The problem in the backtesting leads to unrealistic performance estimates and suboptimal trading strategies, costing institutional investors millions in execution costs.

The results in Chapter~\ref{chapter:experiments} demonstrate that our framework has good alignment with real FX spot market in terms of market conditions, temporal dependencies, clustering properties and statistical distributions of aggressive trades. The XGBoost filter effectively handles the severe class imbalance problem where aggressive trades represent less than 1\% of all market events. The neural Hawkes process then captures the temporal dependencies and clustering patterns that characterize real aggressive trading behavior. Together, these components give realistic and dynamic predictions, which are two critical requirements confirmed by evaluation metrics results.

% [Comparison with Existing Approaches, simplications for FX backtesting environments]
Compared to the existing backtesting approach at MN as shown in Table~\ref{tab:filling_spread}, which uses static filling probabilities based only on bid-ask spreads, the framework provides realistic and dynamic interpretability through the Hawkes process intensity function, showing why and when aggressive trades occur. Traditional methods ignore the clustering of trades and dynamic market conditions, leading to unrealistic execution scenarios. For pension funds managing billions of euros like MN, even small improvements in execution quality can save millions in trading costs. By providing more accurate predictions of when passive orders will be filled, our framework helps traders make better decisions about order timing and pricing. This leads to reduced market impact, lower transaction costs, and improved overall performance.  This transparency is crucial for institutional investors who need to understand and justify their trading strategies. Unlike black-box machine learning models, our approach offers both accuracy and explainability, making it practical for real-world backtesting implementation.


% [Research Limitations]
Despite these contributions, our study has several important limitations that should be acknowledged.
%[Limitations of the Research Question]
First, our definition of aggressive trades as a binary classification problem simplifies market reality. I do this simplification because the application in MN backtesting system pay less attention to the level of aggressive trades. But in practice, aggressiveness exists on a continuous spectrum. Some trades may be moderately aggressive, while others are extremely urgent. Future research could explore continuous measures of aggressiveness or multi-class classification approaches.

Second, we focus only on aggressive trades that remove liquidity from the order book. However, other types of market events, such as large passive orders or sudden cancellations, also affect execution quality. More important interactions between different types of market activity can be researched.

%[Methodological Limitations]
Our methodology also has limitations. The XGBoost filter, while effective at handling class imbalance, still lacks interpretability. Explainable Boosting Machine (EBM) from the interpretable machine learning (GlassBox) family might be a good replace of XGBoost. 
The neural Hawkes process assumes that past events influence future events through exponential decay functions. This may not accurately represent all types of market memory. However, deep learning models are not recommended for this purpose, because these models would require much more data and computational resources, and they would lack the interpretability, making them less practical for real-time applications. These reason makes our approach valuable for practitioners.
% The GRU-based neural network, while more efficient than LSTM networks, may still struggle with very long-term dependencies in market data. If aggressive trades are influenced by events that happened hours or days ago, our model might miss these connections.

%[Data Limitations]
Our data comes from only two trading venues (LMAX and CBOE). This raises questions about the generalizability of our results. Different venues have different market structures, participant types, and trading rules. Our model may not perform as well on other venues or in different market conditions.

% The data also has quality issues common to high-frequency financial data. Some trades appear as "hidden orders" that do not show up in the order book, making it difficult to build a complete picture of market activity. Missing data, delayed reporting, and measurement errors may affect our results.

% Furthermore, our training data covers only normal market conditions. During periods of high volatility, market stress, or major news events, trading patterns may change dramatically. Our model may not perform well during these exceptional periods, which are often the most important times for risk management.

% [Broader Applications and Future Research]
Beyond FX markets, our framework can be applied to other high-frequency trading environments such as equity and cryptocurrency markets. The combination of class imbalance handling and temporal modeling makes it suitable for any domain where rare events cluster in time and influence future occurrences. Outside finance, the framework could be valuable for medical diagnosis of rare diseases, fraud detection in banking, network intrusion detection in cybersecurity, and failure prediction in industrial systems. In all these cases, events are rare, temporally dependent, and require models that capture both sequence patterns and intensity dynamics.

% [Future Research Directions]
Several promising directions for future research could lie in multi-event modeling, multi-asset and cross-market analysis, and Enhanced feature engineering. Instead of focusing only on aggressive trades, future models could predict multiple types of market events. This would provide a more complete picture of market dynamics and potentially improve prediction accuracy through cross-event dependencies. Additionally, extending the framework to handle multiple currency pairs or different asset classes could improve strategy performance across diverse portfolios. Moreover, incorporating additional data sources such as news sentiment, economic indicators, or social media activity could improve prediction accuracy. 
%Machine learning techniques for automatic feature selection could help identify the most valuable predictors.
% Alternative model architectures: Exploring different neural network architectures, such as transformer models or graph neural networks, might capture market relationships that our current approach misses.
% Real-time implementation: Developing efficient algorithms for real-time prediction and investigating the computational trade-offs between model complexity and prediction speed would make the framework more practical for live trading.
% Robustness testing: Systematic evaluation of model performance across different market conditions, venues, and time periods would help establish the limits of the approach and identify when alternative methods might be needed.

In conclusion, our framework represents a significant step forward in modeling aggressive trades for backtesting purposes, and it also opens up many new questions and opportunities for future research. The combination of practical impact and scientific rigor makes this an exciting area for continued investigation.
